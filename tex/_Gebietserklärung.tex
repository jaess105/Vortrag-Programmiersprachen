\subsection{Gebiete der Softwareentwicklung}

\newSectionPage{\insertsubsectionhead}

\begin{frame}{Gebiete der Softwareentwicklung}
    \begin{itemize}
        % Zu viele Gebiete. Können nicht alle gelistet werden
        \item Web development (Webentwicklung)
        \item Application development (Anwendungsentwicklung)
        \item Data science
        \item Artificial Intelligence $\supsetneq$ Machine Learning
              (künstliche Intelligenz $\supsetneq$ maschinelles Lernen)
        \item Game development (Spieleentwicklung)
        \item Embedded systems (eingebettete Systeme)
        \item Operating systems (Betriebssysteme)
    \end{itemize}

    \note{
        \begin{itemize}
            \item nur übersicht
            \item Werden auf alles eingehen
            \item keine große Erklärung hier
        \end{itemize}
    }
\end{frame}

\begin{frame}{Web- \& App development}
    \begin{itemize}
        \item Entwicklung für Enduser
        \item Webentwicklung für den Browser bzw. das Internet
        \item Applikationsentwicklung für ein oder mehrere Betriebssysteme
        \item Die Ziele sind gleich oder ähneln sich
        \item Cross-platform development für eine Code base für beide Bereiche
    \end{itemize}

    \note{
        \begin{itemize}
            \item geht um: Anwendungen für user
            \item meisten direkter Kontakt
            \item Amazon Beispiel:
                  \begin{itemize}
                      \item Webseite -> Webentwicklung:
                            Browser spezifisch
                      \item App -> Applikationsentwicklung:
                            platform spezifisch
                      \item Platformen: Android,  iOS,  Windows,  Linux
                            und MacOs
                  \end{itemize}
            \item Ziele sind gleich, Mittel sind anders
            \item Bereiche waren immer getrennt in letzten Jahren Cross-Platform Entwicklung
            \item Cross Platform:
                  \begin{itemize}
                      \item Eine Code base für alles
                      \item Realität, spezifischer Code
                  \end{itemize}
        \end{itemize}
    }
\end{frame}

\begin{frame}{Data scienece}
    \begin{itemize}
        \item Interdisziplinäres Feld
        \item Wissen aus Daten gewinnen
    \end{itemize}

    \note{
        \begin{itemize}
            \item interdisziplinär
            \item Ziel: Wissen aus Daten zu gewinnen
            \item Methoden aus: Mathematik, Statistik und Informatik
            \item Heute: Auch Maschinelles Lernen
            \item Beispiel:
                  \begin{itemize}
                      \item Kundensegmentierung im Marketing
                      \item Kunden in Gruppen unterteilt -> ähnliches Kaufverhalten
                      \item Vorschläge basierend darauf
                  \end{itemize}
        \end{itemize}
        % Data  science,  ist  ein  interdisziplinäres  Feld  indem  es  darum  geht  Wissen  aus  Daten  zu  gewinnen.
        % Hierbei werden verschiedenste Methoden aus der Mathematik, Statistik und Informatik verwendet.
        % Heutzutage sind einige Methoden zur Informationsgewinnung auch gestützt durch Machine
        % Learning Algorithmen.
        % Ein Beispiel für eine Data science Anwendung ist Kundensegmentierung im Marketing. Hier
        % werden  Kunden  in  Gruppen  unterschieden,  die  ähnliches  Kaufverhalten  haben  um  basierend
        % darauf  Vorschläge  zu  machen,  welche  Produkte  ihnen  gefallen  könnten.  Data  science  wird  hier
        % verwendet um basierend auf den vorhandenen Daten der Kunden Gruppen zu bilden.
    }
\end{frame}

\begin{frame}{Artificial Intelligence $\supsetneq$ Machine Learning}
    \begin{itemize}
        \item Viele Konzepte und Algorithmen inbegriffen
        \item Maschinelles Lernen ist ein Teilbereich
        \item Verwendet in: Alexa, Siri, ...
        \item Viele Firmen sind interessiert in Algorithmen dieser Art
    \end{itemize}

    \note{
        \begin{itemize}
            \item Viele Konzepte und Teilbereich umfasst
            \item maschinelles lernen sehr gehypet
            \item Hauptziel: intelligentes Verhalten automatisieren
            \item Mittel: Algorithmen und Computer
            \item Beispiel:
                  \begin{itemize}
                      \item Alexa, Siri, Google Assistent, Webseite Akinator, GitHub Copilot
                  \end{itemize}
            \item Viele Firmen wollen dies
            \item Ob sinnvoll -> anderer Vortrag
        \end{itemize}
        % Der Bereich der Künstlichen Intelligenz umfasst sehr viele Konzepte und teil Bereiche. Momentan,
        % ist  vor  allem  der  Teilbereich  des  Machine  Learnings  oder  Maschinellen  Lernens  für  viele  sehr
        % interessant.  Das  Hauptziel  des  Bereiches  ist  es  intelligentes  Verhalten  zu  automatisieren  und
        % mithilfe von Computern und Algorithmen dar zu stellen. Wir alle haben in irgendeiner Form schon
        % mit  Algorithmen,  die  unter  den  Bereich  der  Künstlichen  Intelligenz  fallen  Kontakt  gehabt.  Sei  es
        % Alexa, Siri, Google Assistent, die Webseite Akinator (falls ihr die noch kennt), GitHub Copilot, etc..
        % Viele Firmen wollen jetzt Algorithmen aus dem Bereich der Künstlichen Intelligenz nutzen, weil es
        % ein  sehr  vielversprechender  und  interessanter  Zweig  der  Informatik  ist,  den  viele  Große  Firmen
        % verwenden.  Dabei  stellt  sich  manchmal  die  Frage,  ob  das  wirklich  notwendig  ist,  aber  das  ist  eine
        % Frage für einen anderen Vortrag.
    }
\end{frame}

\begin{frame}{Game development}
    \begin{itemize}
        \item Viele Sprachen
        \item Sehr beliebt
        \item Nicht die beste Branche was Bezahlung angeht
    \end{itemize}

    \note{
        \begin{itemize}
            \item Sehr viele Sprachen
            \item Interesse bei Zuhörern
            \item VR... mehr Möglichkeiten
            \item Leider Arbeitsverhältnisse nicht optimal
        \end{itemize}
        % Im  Bereich  der  Spieleentwicklung  gibt  es  auch  unterschiedliche  Sprachen  die  verwendet  werden.
        % Es ist ein Bereich, der wahrscheinlich viele von euch interessiert. Durch VR und ähnliche
        % Entwicklungen  in  den  letzten  Jahren,  sind  die  Möglichkeiten,  was  in  dieser  Branche  möglich  ist
        % nochmal  gestiegen.  Leider  sind  in  der  Spieleentwicklung  die  Anforderungen  sehr  hoch  und  die
        % Bezahlung im Vergleich zu anderen Bereichen der Softwareentwicklung sher schlecht.
    }
\end{frame}

\begin{frame}{Embedded systems}
    \begin{itemize}
        \item Unsichtbare Aufgaben
        \item Abdecken von verschiedensten Aufgaben
        \item Vernetzen von verschiedenen autonomen Systemen
    \end{itemize}

    \note{
        \begin{itemize}
            \item Benutzer unsichtbare Aufgaben
            \item verschiedene Aufgaben
            \item in Großen Systemen:
                  \begin{itemize}
                      \item Überwachungs-,  Steuerungs-,  Regelfunktionen
                      \item Daten-  bzw.  Signalverarbeitung
                  \end{itemize}
            \item Häufig viele autonome vernetzt
            \item Beispiel GPS:
                  \begin{itemize}
                      \item verwendet Satelliten und Empfänger
                      \item Sync: Ort, Zeit, Geschwindigkeit
                  \end{itemize}
                  % Das GPS ist ein Navigationssystem, das Satelliten und Empfänger verwendet,
                  %  um Daten in Bezug auf Ort, Zeit und Geschwindigkeit zu synchronisieren. 
                  %  Der Empfänger oder das Gerät, das die Daten empfängt, hat ein integriertes 
                  %  eingebettetes System, um die Anwendung eines globalen Positionierungssystems
                  %   zu erleichtern. Die eingebetteten GPS-Geräte ermöglichen es den Menschen, ihre 
                  %   aktuellen Standorte und Ziele leicht zu finden. Daher gewinnen sie schnell an 
                  %   Bedeutung und werden zu den am häufigsten verwendeten Navigationsgeräten für 
                  %   Automobile.

        \end{itemize}
        % In  eingebetteten  Systemen  werden  häufig  für  den  Benutzer  unsichtbare  Aufgaben  verrichtet.
        % Eingebettete  Systeme  übernehmen  hier  verschiedene  Aufgaben  in  einem  Großen  System,  wie
        % Überwachungs-,  Steuerungs-  oder  Regelfunktionen  oder  sind  für  Daten-  bzw.  Signalverarbeitung
        % zuständig.
        % Beispiele, wo eingebettete Systeme zum Einsatz kommen sind Flugzeuge, Kraftfahrzeuge,
        % Kühlschränke,  Fernseher,  DVD-Player  und  vieles  mehr.  Häufig  handelt  es  sich  bei  eingebetteten
        % Systemen um eine Vielzahl von ansonsten autonomen Systemen, die miteinander Vernetzt werden.
    }
\end{frame}

\begin{frame}{Operating systems}
    \begin{itemize}
        \item Management von Systemressourcen
        \item Schnittstelle zwischen Hardware und Software
        \item Beispiele: Windows, Linux, MacOS, Android, iOS, ...
    \end{itemize}

    \note{
        \begin{itemize}
            \item Zusammenstellung von Programmen:
            \item Managed:
                  \begin{itemize}
                      \item Systemressourcen:
                            Arbeitsspeicher, Festplatten, Ein- und Ausgabegeräte
                  \end{itemize}
            \item Stellt zur Verfügung
            \item Schnittstelle: Hardware Komponenten -> Anwendungssoftware
            \item Abstraktion der Hardware
            \item Alles gleich verwendbar
        \end{itemize}
        % Betriebssysteme sind eine Zusammenstellung von Computerprogrammen, die die
        % Systemressourcen eines Computers wie Arbeitsspeicher, Festplatten, Ein- und Ausgabegeräte
        % verwaltet  und  diesen  Anwendungsprogramme  zur  Verfügung  stellen.  Sie  bilden  dadurch  eine
        % Schnittstelle zwischen Hardware Komponenten und der Anwendungssoftware. Sie sind eine
        % Abstraktion auf der unterliegenden Hardware. Durch das Betriebssystem, haben Anwendungen die
        % Möglichkeit die eigentlichen Spezifikationen der unterliegende Hardware zu ignorieren und
        % können mit ihr kommunizieren, als wäre es immer dieselbe.
    }
\end{frame}