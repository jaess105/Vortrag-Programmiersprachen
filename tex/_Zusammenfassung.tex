\newSectionPage{\insertsectionhead}


\begin{frame}{Zusammenfassung des Inhalts}

    \begin{itemize}
        \item 19 Sprachen wurden vorgestellt
        \item Überblick verschafft
        \item Nicht ansatzweise Vollständig
        \item Sprachen aus der Vergangenheit aber auch aktuelle Sprachen wurden ausgelassen
    \end{itemize}

    \note{
        \begin{itemize}
            \item bekannte Sprachen
            \item Literatur, Studium
            \item nicht alle, die ich verwendet habe, nicht alle, die ich kenne

        \end{itemize}
        % In  dieser  Aufstellung  haben  wir  nun  einige  Programmiersprachen  und  die  Bereiche  in  denen  sie
        % benutzt  werden  kennen  gelernt.  
        % Dieser  Vortrag  soll  lediglich  einen  groben  Überblick  über  die
        % Sprachen und ihre Anwendungsbereiche geben.
        % Des  weiteren,  erhebe  ich  keinerlei  Anspruch  auf  Vollständigkeit.  Weder  die  Anwendungsbereiche
        % noch die genannten Programmiersprachen sind ansatzweise vollständig. Viele
        % Programmiersprachen, die vor allem in der Vergangenheit genutzt wurden, als auch Sprachen die
        % heute verwendet werden, wurden hier nicht aufgelistet. 
    }

\end{frame}

\begin{frame}{Was solltet ihr mitnehmen?}

    \begin{itemize}
        \item Jede Sprache hat Vor- und Nachteile
        \item Jede Sprache hat ihren Verwendungszweck
        \item Neues wird täglich erfunden
        \item Neue Sprachen können Erfahrungen Erweitern und eigenes Denken verbessern
    \end{itemize}

    \note{
        % Jede  Sprache  hat  ihre  Vor-  und  Nachteile  und  ihre  Verwendbarkeit  oder  Berechtigung.  Man  sollte
        % sich  immer  darüber  im  klaren  sein,  dass  jede  Sprache  aus  einem  Grund  verwendet  wird  und  aus
        % einem Grund entwickelt wurde. Neue Konzepte werden täglich erfunden oder anders interpretiert.
        % Eine  neue  Sprache  zu  lernen  kann  nur  die  eigene  Erfahrung  erweitern  und  vielleicht  sogar  das
        % eigene Denken verbessern.

    }

\end{frame}

\begin{frame}{Schlusswort}

    \begin{itemize}
        \item Erstmal eine Sprache lernen
        \item Konzepte sind übertragbar
        \item Java ist gut für den Einstieg
    \end{itemize}

    \note{
        \begin{itemize}
            \item Später hilft vlt dieser Überblick
        \end{itemize}
        % Sei das gesagt, so ist es doch auch wichtig, dass man erst Mal eine Sprache lernt und danach nach
        % dem  die  Grundkonzepte  verstanden  wurden  sich  in  neue  hineinstürzt.  In  meinen  Augen  ist  Java
        % eine  gute  erste  Programmiersprache  mit  der  ich  auch  angefangen  habe.  Dennoch  sollte  man
        % nachdem man eine Sprache gut beherrscht auch mal einen Blick auf andere Werfen und vielleicht
        % hilft bei diesem Blick dann auch diese Übersicht.
    }

\end{frame}