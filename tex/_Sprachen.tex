\newSectionPage{\insertsectionhead}

\begin{frame}{Programmiersprachen}
    ''There are only two kinds of languages: the ones people complain about and the ones nobody uses.''
    \cite{10.5555/2543987}
    \note{
        — Bjarne Stroustrup, The C++ Programming Language

        \begin{itemize}
            \item Manche lieben eine Sprache, manche hassen sie
            \item Keine perfekt
            \item alle vor- und nachteile
            \item muss nicht alle können
            \item hilft zu wissen wann wo welche Sprache sinnvoll
            \item Mehr Konzepte bekannt = besser
        \end{itemize}
    }

\end{frame}

\begin{frame}{HTML \& CSS}
    \textWithImages{
        \begin{itemize}
            \item Bilden Grundlagen der Darstellung von Webseiten
            \item Mithilfe von anderen Sprachen auch in der Anwendungsentwicklung genutzt
            \item HTML: markup language und keine richtige Programmiersprache
        \end{itemize}
    }{
        \imgWithSource{resources/logos/CSS3_HTML5}{\cite{web:cssHTML}}{Css und HTML Logo}
    }

    \note{
        \begin{itemize}
            \item Grundlagen webbrowser
            \item HTML stukturierung, CSS schön
            \item Web basiert auf HTML
            \item Hauptsächlich in Web aber auch Anwenugen durch andere Sprachen
            \item HTML keine Programmiersprache
            \item Keine Logik
            \item Dennoch unerlässlich
        \end{itemize}
    }
\end{frame}

\begin{frame}{ JavaScript }
    \textWithImages{
        \begin{itemize}
            \item Steuer verhalten von Webseiten
            \item Eigenschaften:
                  Interpretiert, dynamisch typisiert, keine Typsicherheit, garbage collected, mehrere Paradigmen
            \item Unerlässlich für dynamische Webseiten
            \item Verwendung:
                  Web- und Applikationsentwicklung, Machine Learning, Data science, game development
        \end{itemize}
    }{
        \imgWithSource{resources/logos/js}{\cite{web:js}}{JavaScript Logo}
    }

    \note{
        \begin{itemize}
            \item HTML und CSS aussehen, JS verhalten
            \item interpretiert, dynamisch typisiert, typ unsicher (prototyped),
                  mehrere Paradigmen
            \item 1995 von Netscape für dynamisch HTML
            \item heute in allen HTML Seiten für dynamischen Content
                  und im backend
            \item sehr umstritten, tausende von Frameworks
            \item Ursprung im Web dort immernoch fast unerlässlich für dyn Seiten
            \item Heute,
                  Android, iOS und Corss-Platform,
                  ML, Data Science, game-dev
            \item Viele Anwendungsmöglichkeiten durch Generalität
        \end{itemize}
    }
\end{frame}

\begin{frame}{ TypeScript }
    \textWithImages{
        \begin{itemize}
            \item Superset von JavaScript
            \item Statisch Typisiert und Typsicher
            \item Verwendung eines Transpilers
            \item Verwendung:
                  Kann fast überall verwendet werden, wo JavaScript verwendet wird
        \end{itemize}
    }{
        \imgWithSource{resources/logos/ts}{\cite{web:ts}}{TypeScript Logo}
    }

    \note{
        \begin{itemize}
            \item Superset
            \item wichtiges Feature, statisch Typisiert und Typsicherheit
                  -> Frühe Erkennung von Bugs
            \item Transpiler -> Verwendung von neuen Features in älterem JS
            \item alle Bereiche, wo JS verwendet wird
        \end{itemize}
    }
\end{frame}

\begin{frame}{PHP}
    \textWithImages{
        \begin{itemize}
            \item Eigenschaften:
                  Interpretiert, dynamisch typisiert, garbage collected, mehrere Paradigmen
            \item Serverseitiges rendern
            \item Verwendung: Webentwicklung
            \item Grundlage für Wordpress und Wikipedia
        \end{itemize}
    }{
        \imgWithSource{resources/logos/php}{\cite{web:php}}{Php Logo}
    }

    \note{
        \begin{itemize}
            \item dynamisch typisiert, interpretiert, GC, viele paradigmen
            \item Gegensatz JS: auf Server gerendert nicht dynamisch im Browser
            \item Nur im Web:
                  Wordpress, Wikipedia
        \end{itemize}
        % Ist eine dynamisch typisierte, interpretierte Sprache, mit einem garbage collector und
        % Unterstützung für viele Paradigmen. PHP wird hauptsächlich zur Erstellung von Webseiten
        % verwendet.  Im  Gegensatz  zu  JavaScript  wird  die  Sprache  nicht  direkt  im  Browser  ausgeführt
        % sondern vom Server, welcher dann eine veränderte Seite basierend auf dem PHP Code an den User
        % schickt.
        % PHP wird ausschließlich in der Webentwicklung verwendet und betreibt Frameworks wie
        % Wordpress und Internetseiten, wie Wikipedia.
    }
\end{frame}

\begin{frame}{Ruby}
    \textWithImages{
        \begin{itemize}
            \item Eigenschaften:
                  Interpretiert, dynamisch typisiert, garbage collected, objektorientiert
            \item Verwendung:
                  \begin{itemize}
                      \item  Webentwicklung mit Ruby on Rails
                      \item Skripte auf Linux, Windows, iOS
                      \item Vereinzelt auch Spieleentwicklung
                  \end{itemize}

        \end{itemize}
    }{
        \imgWithSource{resources/logos/ruby}{\cite{web:ruby}}{Ruby Logo}
    }

    \note{
        \begin{itemize}
            \item Interpretiert, dynamisch typisiert, garbage collected, objektorientiert
            \item unterstützt auch andere Paradigmen
            \item Vorallem in Webentwicklung -> GitHub
            \item Für Skripte: Linux, Windows, iOS
            \item Manchmal Spieleentwicklung
        \end{itemize}
        % Ruby  ist  eine  interpretierte,  dynamisch  Typisierte  Sprache,  mit  einem  garbage  collector.  Ruby
        % wurde  hauptsächlich  für  Objekt  orientierte  Programmierung  entwickelt,  unterstützt  aber  auch
        % andere Paradigmen.
        % Ruby  findet  vor  allem  Verwendung  in  der  Webentwicklung  mit  dem  Framework  Ruby  on  Rails,
        % welches bekannte Webseiten wie GitHub betreibt. Ruby wird auch als Skriptsprache verwendet um
        % kleinere  Aufgaben  auf  Linux,  Windows  oder  iOS  zu  erfüllen.  Vereinzelt  wird  Ruby  auch  in  der
        % Entwicklung von Spielen verwendet.
    }
\end{frame}

\begin{frame}{Java und die JVM}
    \textWithImages{
        \begin{itemize}
            \item Eigenschaften:
                  Kompiliert, statisch typisiert, typsicher, garbage collected, objektorientiert
            \item Plattformunabhängigkeit dank JVM
            \item Grundlage von Android Apps
            \item Verwendung:
                  Web- und Applikationsentwicklung, data science, künstliche Intelligenz, Spieleentwicklung (Minecraft)
            \item JVM ermöglicht Nutzung in anderen Sprachen: Kotlin, Clojure, Scala
        \end{itemize}
    }{
        \imgWithSource{resources/logos/java}{\cite{web:java}}{Java Logo}
    }

    \note{
        \begin{itemize}
            \item Kompiliert, statisch typisiert, typsicher, garbage collected, objektorientiert
            \item JVM
            \item Grundlage für Android
            \item Web- und Applikationsentwicklung, data science, künstliche Intelligenz, Spieleentwicklung (Minecraft)
            \item JVM -> sichert weitere Verwendung
        \end{itemize}
    }
\end{frame}

\begin{frame}{Kotlin}
    \textWithImages{
        \begin{itemize}
            \item Basiert auf der JVM
            \item Kann Java code nutzen
            \item neue Funktionalitäten im Vergleich zu Java
            \item Verwendung:
                  \begin{itemize}
                      \item Überall, wo Java verwendet wird
                      \item Beliebteste Sprache für Android Applikationsentwicklung
                  \end{itemize}
        \end{itemize}
    }{
        \imgWithSource{resources/logos/kotlin}{\cite{web:kotlin}}{Kotlin Logo}
    }

    \note{
        \begin{itemize}
            \item Ähnlich wie Java
            \item deutlich jünger
            \item arbeite auf JVM
            \item interop
            \item Weniger fokussiert auf OOP
            \item neue Funktionalitäten
            \item Kotlin fast überall wo Java verwendet wird
            \item Beliebteste Sprache für Android
        \end{itemize}
    }
\end{frame}
\begin{frame}{Dart}
    \textWithImages{
        \begin{itemize}
            \item Eigenschaften:
                  Kompiliert, statisch oder dynamisch typisiert, garbage collected, objektorientiert
            \item Von google als JavaScript Nachfolger entwickelt
            \item Verschiedene Kompilierungsziele
            \item Verwendung:
                  \begin{itemize}
                      \item Cross-platform Entwicklung mit Flutter
                      \item Noch nicht viel Verwendung in anderen Gebieten
                  \end{itemize}
        \end{itemize}
    }{
        \imgWithSource{resources/logos/dart}{\cite{web:dart}}{Dart Logo}
    }

    \note{
        \begin{itemize}
            \item OOP Unterstützt andere Paradigmen
            \item unterschiedliche Schlüsselwörter -> dynamisch oder statisch
            \item GC
            \item von Google Nachfolger für JS
            \item Kann zu Maschinencode oder JS kompiliert werden
            \item Vor allem in Cross-platform
            \item Nicht viel in anderen Gebiten aber Möglichkeit besteht
        \end{itemize}
    }

\end{frame}

\begin{frame}{Golang}
    \textWithImages{
        \begin{itemize}
            \item Eigenschaften:
                  Kompiliert, statisch oder dynamisch typisiert, garbage collected, prozedural
            \item Benutzt von Google, Uber und Twitch
            \item Entwickelt von Google
            \item Verwendung:
                  \begin{itemize}
                      \item Infrastruktur für Webanwendungen
                      \item Applikationsentwicklung
                      \item Künstliche Intelligenz
                      \item Eingebettete Systeme
                      \item Spieleentwicklung
                  \end{itemize}
        \end{itemize}
    }{
        \imgWithSource{resources/logos/go}{\cite{web:golang}}{Golang Logo}
        % \imgWithSource{resources/logos/go-mascot.png}{\cite{web:go-mascot}}{Golang Maskottchen ''Gopher''}
    }

    \note{
        \begin{itemize}
            \item Golang kurz Go, weil es um Geschwindigkeit geht
            \item Kompiliert, statisch oder dynamisch typisiert, garbage collected, prozedural
            \item Entwickelt von Google
            \item Verwendet: Google, Uber und Twitch
            \item Gute Performance und Lesbarkeit
            \item Viel für Infrastruktur verwendet
            \item Aber sonst auch in fast allen Gebieten
            \item GC aber wenig Speicher daher auch Eingebettete Systeme
        \end{itemize}
    }
\end{frame}

\begin{frame}{Sprachen mit manueller Speicherallokation}
    \begin{itemize}
        \item Drei viel benutzte Sprachen mit manuellem Speichermanagement
        \item C, C++, Rust
        \item Verwendung:
              \begin{itemize}
                  \item Eingebettete Systeme
                  \item Aber auch in anderen Bereichen verwendet
              \end{itemize}
        \item Vorteil in Laufzeiten
    \end{itemize}

    \note{
        \begin{itemize}
            \item Am häufigsten verwendet mit man. Speicherallokation
            \item Hauptsächlich eingebettete Systeme
        \end{itemize}
    }

\end{frame}

\begin{frame}{C}
    \textWithImages{
        \begin{itemize}
            \item Älteste Sprache
            \item Nur Unterstützung von Prozeduraler Programmierung
            \item Grundlage für Sprachen wie: Java, Python, Objective-C, C++, C\#
            \item Grundlage für die meisten Betriebssysteme
        \end{itemize}
    }{
        \imgWithSource{resources/logos/c}{\cite{web:C}}{C Logo}
    }

    \note{
        \begin{itemize}
            \item weit verbreitet
            \item viele können sie
        \end{itemize}
    }

\end{frame}

\begin{frame}{C++}
    \textWithImages{
        \begin{itemize}
            \item Superset von C mit wenigen Ausnahmen
            \item Unterstützung von objektorientierter Programmierung
            \item Verwendung:
                  \begin{itemize}
                      \item Eingebettete Systeme
                      \item Web- und Applikationsentwicklung
                      \item Spieleentwicklung, Grundlage für Unreal engine
                  \end{itemize}
        \end{itemize}
    }{
        \imgWithSource{resources/logos/cpp}{\cite{web:cpp}}{C++ Logo}
    }

    \note{
        \begin{itemize}
            \item unterschied zwischen C und C++ OOP
            \item Neue features
            \item auch Funktional
            \item Sehr kompliziert wahrgenommen
        \end{itemize}
    }
\end{frame}

\begin{frame}{Rust}
    \textWithImages{
        \begin{itemize}
            \item Neuste von den Drei Programmiersprachen
            \item Integration von neuen Konzepten
            \item Große Beliebtheit
        \end{itemize}
    }{
        \imgWithSource{resources/logos/rust}{\cite{web:rust}}{Rust Logo}
        \imgWithSource{resources/logos/rust-mascot}{\cite{web:rust-mascot-feris}}{Rust Maskottchen ''Feris''}
    }

    \note{
        \begin{itemize}
            \item mehere Paradigmen
            \item neue Konzepte: Owndership Konzept, Null safety
            \item Einfacher Speicher nachdenken
                  -> leichtere Bug erkennung
            \item Vor allem Eingebettete Systeme
        \end{itemize}
    }
\end{frame}


\begin{frame}{C\# und das .NET Framework}
    \textWithImages{
        \begin{itemize}
            \item Ähnelt eher Java als C++
            \item Eigenschaften:
                  Kompiliert, statisch typisiert, typsicher, garbage collected,
                  objektorientiert
            \item Gestützt durch das .NET Framework
            \item Grundlage für viele Windows Applikationen
            \item Interaktion mit andere Sprachen durch .NET
        \end{itemize}
    }{
        \imgWithSource{resources/logos/csharp}{\cite{web:csharp}}{C\# Logo}
    }
    \note{
        \begin{itemize}
            \item Keine direkte Weiterentwicklung von C++
            \item Kompiliert, statisch typisiert, typsicher, garbage collected,
                  objektorientiert
            \item .NET Framework bietet Interop mit anderen Sprachen und Microsoft support
            \item Anfänglich nur Windows dadurch viel in Windows
            \item Wie Java viele Bereiche aber Hauptsächlich: Web, App und Spieleentwicklung
            \item Grundlage von Unity engine
        \end{itemize}
    }
\end{frame}

\begin{frame}{Python}
    \textWithImages{
        \begin{itemize}
            \item Sehr beliebte Sprache
            \item Einfache Syntax
            \item Eigenschaften:
                  Interpretiert, dynamisch typisiert, garbage collected,
                  mehrere Paradigmen
            \item Verwendung in fast allen Bereichen
            \item Beliebteste Sprache für machine Learning
            \item Langsame Laufzeit
        \end{itemize}
    }{
        \imgWithSource{resources/logos/python}{\cite{web:py}}{Python Logo}
    }

    \note{
        \begin{itemize}
            \item Darf niemals fehlen
            \item beliebt unf einfach
            \item viele Bibliotheken
            \item Überall verwendet vor allem ML
            \item Auch in dynamischen Webseiten
            \item Nicht in eingebetteten Systemen
        \end{itemize}
    }

\end{frame}
\begin{frame}{Objective-C und Swift}
    \textWithImages{
        \begin{itemize}
            \item Apples Sprachen
            \item Kaum verwendet in anderen Bereichen
        \end{itemize}
    }{
        \imgWithSource{resources/logos/objective-c.png}{\cite{web:Objective-c}}{Inoffizielles Objective-C Logo}
        \imgWithSource{resources/logos/swift.png}{\cite{web:swift}}{Swift Logo}
    }

    \note{
        \begin{itemize}
            \item Hauptsächlich/ausschließlich in Apple Öko
            \item Swift ist Nachfolger von Objective-C
            \item Swift ist modern und bietet neue Features
        \end{itemize}
    }

\end{frame}

\begin{frame}{R \& Matlab}
    \textWithImages{
        \begin{itemize}
            \item Domänen spezifische Sprachen
            \item R in Data science und Statistik
            \item Matlab für mathematische Probleme
            \item Kaum Verwendung außerhalb ihrer Domäne
        \end{itemize}
    }{
        \imgWithSource{resources/logos/r.png}{\cite{web:R}}{R Logo}
        \imgWithSource{resources/logos/matlab.png}{\cite{web:matlab}}{Matlab Logo}
    }

    \note{
        \begin{itemize}
            \item letzte
            \item nicht viel gemein außer domain spezifisch
            \item R in Data Science und Statistik
            \item Matlab in Mathe
            \item Können auch andere Sachen, werden dafür aber nicht viel verwendet
        \end{itemize}
    }
\end{frame}