\begin{frame}{Programmiersprachen}
    ''There are only two kinds of languages: the ones people complain about and the ones nobody uses.''
    — Bjarne Stroustrup, The C++ Programming Language

\end{frame}

\begin{frame}{HTML \& CSS}
    \textWithImages{
        \begin{itemize}
            \item Bilden Grundlagen der Darstellung von Webseiten
            \item Mithilfe von anderen Sprachen auch in der Anwendungsentwicklung genutzt
            \item HTML: markup language und keine richtige Programmiersprache
        \end{itemize}
    }{
        \imgWithSource{resources/logos/CSS3_HTML5}{}
    }
\end{frame}

\begin{frame}{ JavaScript }
    \textWithImages{
        \begin{itemize}
            \item Steuer verhalten von Webseiten
            \item Eigenschaften:
                  Interpretiert, dynamisch typisiert, keine Typsicherheit, garbage collected, mehrere Paradigmen
            \item Unerlässlich für dynamische Webseiten
            \item Verwendung:
                  Web- und Applikationsentwicklung, Machine Learning, Data science, game development
        \end{itemize}
    }{
        \imgWithSource{resources/logos/js}{}
    }

\end{frame}

\begin{frame}{ TypeScript }
    \textWithImages{
        \begin{itemize}
            \item Superset von JavaScript
            \item Statisch Typisiert und Typsicher
            \item Verwendung eines Transpilers
            \item Verwendung:
                  Kann fast überall verwendet werden, wo JavaScript verwendet wird
        \end{itemize}
    }{
        \imgWithSource{resources/logos/ts}{}
    }
\end{frame}

\begin{frame}{PHP}
    \textWithImages{
        \begin{itemize}
            \item Eigenschaften:
                  Interpretiert, dynamisch typisiert, garbage collected, mehrere Paradigmen
            \item Serverseitiges rendern
            \item Verwendung: Webentwicklung
            \item Grundlage für Wordpress und Wikipedia
        \end{itemize}
    }{
        \imgWithSource{resources/logos/php}{}
    }
\end{frame}

\begin{frame}{Ruby}
    \textWithImages{
        \begin{itemize}
            \item Eigenschaften:
                  Interpretiert, dynamisch typisiert, garbage collected, objektorientiert
            \item Verwendung:
                  \begin{itemize}
                      \item  Webentwicklung mit Ruby on Rails
                      \item Skripte auf Linux, Windows, iOS
                      \item Vereinzelt auch Spieleentwicklung
                  \end{itemize}

        \end{itemize}
    }{
        \imgWithSource{resources/logos/ruby}{}
    }
\end{frame}

\begin{frame}{Java und die JVM}
    \textWithImages{
        \begin{itemize}
            \item Eigenschaften:
                  Kompiliert, statisch typisiert, typsicher, garbage collected, objektorientiert
            \item Plattformunabhängigkeit dank JVM
            \item Grundlage von Android Apps
            \item Verwendung:
                  Web- und Applikationsentwicklung, data science, künstliche Intelligenz, Spieleentwicklung (Minecraft)
            \item JVM ermöglicht Nutzung in anderen Sprachen: Kotlin, Clojure, Scala
        \end{itemize}
    }{
        \imgWithSource{resources/logos/java}{}
    }
\end{frame}

\begin{frame}{Kotlin}
    \textWithImages{
        \begin{itemize}
            \item Basiert auf der JVM
            \item Kann Java code nutzen
            \item neue Funktionalitäten im Vergleich zu Java
            \item Verwendung:
                  \begin{itemize}
                      \item Überall, wo Java verwendet wird
                      \item Beliebteste Sprache für Android Applikationsentwicklung
                  \end{itemize}
        \end{itemize}
    }{
        \imgWithSource{resources/logos/kotlin}{}
    }

\end{frame}
\begin{frame}{Dart}
    \textWithImages{
        \begin{itemize}
            \item Eigenschaften:
                  Kompiliert, statisch oder dynamisch typisiert, garbage collected, objektorientiert
            \item Von google als JavaScript Nachfolger entwickelt
            \item Verschiedene Kompilierungsziele
            \item Verwendung:
                  \begin{itemize}
                      \item Cross-platform Entwicklung mit Flutter
                      \item Noch nicht viel Verwendung in anderen Gebieten
                  \end{itemize}
        \end{itemize}
    }{
        \imgWithSource{resources/logos/dart}{}
    }

\end{frame}

\begin{frame}{Golang}
    \textWithImages{
        \begin{itemize}
            \item Eigenschaften:
                  Kompiliert, statisch oder dynamisch typisiert, garbage collected, prozedural
            \item Benutzt von Google, Uber und Twitch
            \item Entwickelt von Google
            \item Verwendung:
                  \begin{itemize}
                      \item Infrastruktur für Webanwendungen
                      \item Applikationsentwicklung
                      \item Künstliche Intelligenz
                      \item Eingebettete Systeme
                      \item Spieleentwicklung
                  \end{itemize}
        \end{itemize}
    }{
        \imgWithSource{resources/logos/go}{}
    }
    % \textWithImages{Content}{resources/logos/go-mascot.png}

\end{frame}

\begin{frame}{Sprachen mit manueller Speicherallokation}

    \begin{itemize}
        \item Drei viel benutzte Sprachen mit manuellem Speichermanagement
        \item C, C++, Rust
        \item Verwendung:
              \begin{itemize}
                  \item Eingebettete Systeme
                  \item Aber auch in anderen Bereichen verwendet
              \end{itemize}
        \item Vorteil in Laufzeiten
    \end{itemize}

\end{frame}

\begin{frame}{C}
    \textWithImages{
        \begin{itemize}
            \item Älteste Sprache
            \item Nur Unterstützung von Prozeduraler Programmierung
            \item Grundlage für Sprachen wie: Java, Python, Objective-C, C++, C\#
            \item Grundlage für die meisten Betriebssysteme
        \end{itemize}
    }{
        \imgWithSource{resources/logos/c}{}
    }

\end{frame}

\begin{frame}{C++}
    \textWithImages{
        \begin{itemize}
            \item Superset von C mit wenigen Ausnahmen
            \item Unterstützung von objektorientierter Programmierung
            \item Verwendung:
                  \begin{itemize}
                      \item Eingebettete Systeme
                      \item Web- und Applikationsentwicklung
                      \item Spieleentwicklung, Grundlage für Unreal engine
                  \end{itemize}
        \end{itemize}
    }{
        \imgWithSource{resources/logos/cpp}{}
    }

\end{frame}

\begin{frame}{Rust}
    \textWithImages{
        \begin{itemize}
            \item Neuste von den Drei Programmiersprachen
            \item Integration von neuen Konzepten
            \item Große Beliebtheit
        \end{itemize}
    }{
        \imgWithSource{resources/logos/rust}{}
        \imgWithSource{resources/logos/rust-mascot}{}
    }

\end{frame}


\begin{frame}{C\# und das .NET Framework}
    \textWithImages{
        \begin{itemize}
            \item Ähnelt eher Java als C++
            \item Eigenschaften:
                  Kompiliert, statisch typisiert, typsicher, garbage collected,
                  objektorientiert
            \item Gestützt durch das .NET Framework
            \item Grundlage für viele Windows Applikationen
            \item Interaktion mit andere Sprachen durch .NET
        \end{itemize}
    }{
        \imgWithSource{resources/logos/csharp}{}
    }

\end{frame}

\begin{frame}{Python}
    \textWithImages{
        \begin{itemize}
            \item Sehr beliebte Sprache
            \item Einfache Syntax
            \item Eigenschaften:
                  Interpretiert, dynamisch typisiert, garbage collected,
                  mehrere Paradigmen
            \item Verwendung in fast allen Bereichen
            \item Beliebteste Sprache für machine Learning
            \item Langsame Laufzeit
        \end{itemize}
    }{
        \imgWithSource{resources/logos/python}{}
    }

\end{frame}
\begin{frame}{Objective-C und Swift}
    \textWithImages{
        \begin{itemize}
            \item Apples Sprachen
            \item Kaum verwendet in anderen Bereichen
        \end{itemize}
    }{
        \imgWithSource{resources/logos/objective-c.png}{}
        \imgWithSource{resources/logos/swift.png}{}
    }

\end{frame}

\begin{frame}{R \& Matlab}
    \textWithImages{
        \begin{itemize}
            \item Domänen spezifische Sprachen
            \item R in Data science und Statistik
            \item Matlab für mathematische Probleme
            \item Kaum Verwendung außerhalb ihrer Domäne
        \end{itemize}
    }{
        \imgWithSource{resources/logos/r.png}{}
        \imgWithSource{resources/logos/matlab.png}{}
    }
\end{frame}