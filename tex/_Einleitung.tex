% Überlegung. Wenn Programmiersprachen aufgelistet werden und dann die Gebiete unter die sie fallen,
% dann verliert man den Überblick welche Programmiersprachen für welche Gebiete verwendet werden können.
% Wenn die Gebiete und dann die Programmiersprachen, die für sie infrage kommen aufgelistet werden, 
% dann wird der Fokus von den Programmiersprachen wegbewegt und eher zu den Gebieten hinbewegt.
% Proposal:
% 1. Eine Auflistung von allen gebieten und was ihre Rolle ist.
% 2. Facetten von Programmiersprachen Einführung (Paradigmen, Typsicherheit...)
% 3. Durchgehen der verschiedenen Programmiersprachen. Listen wofür sie sinnvoll sind.
% 4. Zusammenfassung der Gebiete und welche Programmiersprachen für sie relevant sind
%    als Abschluss mit schnellem durchgehen der Folien zum späteren Nachlesen bei Interesse.
%
% Notes:
% - Vortrag über Quantum Computing in den Cybersecurity Teil einbauen https://www.youtube.com/watch?v=D1RCj_g5amA&t=1970s
% - CLIPS als Beispiel einer hoch spezialisierten Sprache.
% - Mica Fragen, was für Programmiersprachen er so benutzt abgesehen von Python.

% INHALTSVERZEICHNIS (GGF. ENTFERNEN)
\begin{frame}{Inhalt}
    \thispagestyle{empty}
    \tableofcontents

    \note {
        \begin{itemize}
            \item Willkommen 'Programmiersprachen: Ein genereller Überblick, welche es gibt und wo sie verwendet werden'
            \item angucken:
            \begin{itemize}
                \item gebiete in welchen die Wahl wichtig
                \item Unterschiede eigenschaften von Programmiersprachen
                \item Eigentliche Programmiersprachen
            \end{itemize}
            \item Nichts vollständig -> Sisyphos Aufgabe, kein statisches Gebiet
            \item Fragen zum Vortrag am Ende
        \end{itemize}
    }
\end{frame}