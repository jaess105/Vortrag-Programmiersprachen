\subsection{Unterschiede in Programmiersprachen}

\begin{frame}{Paradigmen}
    \begin{itemize}
        \item Unterschiedliche Arten Funktionalitäten darzustellen
        \item Prozedural, objektorientiert, funktional
    \end{itemize}
\end{frame}



\begin{frame}{Prozedural}
    \begin{itemize}
        \item Älteste Form des strukturierten Programmierens
        \item Programm flow von oben nach unten
        \item Abstraktionen durch Funktionen
        \item Zusammenhalten von Daten durch Strukturen möglich
    \end{itemize}
\end{frame}

\begin{frame}{Objektorientiert}
    \begin{itemize}
        \item Zusammenfassen von Daten als Objekte
        \item Objekte haben Eigenschaften und Verhalten
        \item Prozedurale Eigenschaften vorhanden durch Programmflow von oben nach unten
    \end{itemize}
\end{frame}

\begin{frame}{Funktional}
    \begin{itemize}
        \item Erinnert an prozedurale Programmierung
        \item Zustand von Strukturen wird nicht verändert
        \item Funktionen erstellen Kopien von ursprünglichen Strukturen mit Veränderungen
        \item $x_1 = x_2 \Rightarrow f(x_1) = y \text{ und } f(x_2) = y$
        \item Beliebtheit steigt
    \end{itemize}
\end{frame}

\begin{frame}[fragile]{Typsicherheit}
    \begin{itemize}
        \item Fehler Entdeckung bei inkompatiblen Typen
        \item Typen sind z.B. \inlinecode{int} und \inlinecode{String}
    \end{itemize}
    \begin{javabox}{Typsicherheit}
        \begin{lstlisting}[style=java]
int x = 3;
String str = "Hello There";
System.out.println(str - x); <- Führt zu Kompilierungsfehler
        \end{lstlisting}
    \end{javabox}
\end{frame}

\begin{frame}{Statically vs. dynamically typed}
    \begin{itemize}
        \item Sprachen wie Java
        \item Datentyp wird einmal deklariert und steht fest
        \item Nach \inlinecode{int x = 3;}
              sind \inlinecode{String x = ''Hello'';}
              und \inlinecode{x = ''There'';}
              nicht mehr möglich.
        \item Python: \inlinecode{x = 3} dann \inlinecode{x = ''Ah General Kenobi''} möglich.
    \end{itemize}
\end{frame}

\begin{frame}{Compiled vs. interpreted}
    \begin{itemize}
        \item Java kompiliert durch \inlinecode{javac} Befehl
        \item \inlinecode{.class} Datei ist Maschinencode aus \inlinecode{.java} Datei
        \item Interpretierte Sprachen: JavaScript, Python, ...
        \item Kompilierte Sprachen meistens schneller
    \end{itemize}
\end{frame}

\begin{frame}{Garbage collected vs. manuelle Speicherallokation}
    \begin{itemize}
        \item Unterschied im Speichermanagement
        \item Manuelle Speicherallokation wird von Programmierer:in kontrolliert
        \item Garbage collector automatisiert Speicher management
        \item Manuelle Speicherallokation ist schneller und braucht weniger Speicher
              ist aber schwieriger
    \end{itemize}
\end{frame}