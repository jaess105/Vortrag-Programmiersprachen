\input{../vorkurs-2022/Folien/head/HHU_Header}

\title{Programmiersprachen}
\subtitle{Ein genereller Überblick, welche es gibt und wo sie verwendet werden.}


\begin{document}

% TITELFRAME BITTE NICHT ÄNDERN
\backgroundTitle
\begin{frame}
    \thispagestyle{empty}
    \begin{columns}
        \column{0.4\paperwidth}{\footnotesize\color{hhuBlau}\put(20,-200){\insertdate}}
        \column{0.6\paperwidth}
        \color{hhuBlau}
        \LARGE \inserttitle\\[\baselineskip]
        \large \insertauthor
    \end{columns}
\end{frame}
\backgroundNormal

% INHALTSVERZEICHNIS (GGF. ENTFERNEN)
\begin{frame}{Inhalt}
    \thispagestyle{empty}
    \tableofcontents
\end{frame}


\section{Einführung}

% Überlegung. Wenn Programmiersprachen aufgelistet werden und dann die Gebiete unter die sie fallen,
% dann verliert man den Überblick welche Programmiersprachen für welche Gebiete verwendet werden können.
% Wenn die Gebiete und dann die Programmiersprachen, die für sie infrage kommen aufgelistet werden, 
% dann wird der Fokus von den Programmiersprachen wegbewegt und eher zu den Gebieten hinbewegt.
% Proposal:
% 1. Eine Auflistung von allen gebieten und was ihre Rolle ist.
% 2. Facetten von Programmiersprachen Einführung (Paradigmen, Typsicherheit...)
% 3. Durchgehen der verschiedenen Programmiersprachen. Listen wofür sie sinnvoll sind.
% 4. Zusammenfassung der Gebiete und welche Programmiersprachen für sie relevant sind
%    als Abschluss mit schnellem durchgehen der Folien zum späteren Nachlesen bei Interesse.
%
% Notes:
% - Vortrag über Quantum Computing in den Cybersecurity Teil einbauen https://www.youtube.com/watch?v=D1RCj_g5amA&t=1970s
% - CLIPS als Beispiel einer hoch spezialisierten Sprache.
% - Mica Fragen, was für Programmiersprachen er so benutzt abgesehen von Python.



\begin{frame}{Gebiete der Softwareentwicklung}
    \begin{itemize}
        % Zu viele Gebiete. Können nicht alle gelistet werden
        \item Web development
        \item Application development
        \item Databases
        \item Data science
        \item Artificial Intelligence $\supsetneq$ Machine Learning
        \item Cyber Security
        \item Game development
        \item Embedded systems
        \item Operating systems
    \end{itemize}

    \note{
        \begin{itemize}
            \item Erst war die Überschrift Gebiete der Informatik,
                  aber die Informatik umfasst mehr als nur die reine Softwareentwicklung.
            \item Keine Vollständigkeit der Liste.
                  Lediglich ein überblick über die Gebiete, die die Informatik oder vielmehr die Softwareentwicklung abdeckt.
        \end{itemize}
    }
\end{frame}

\section{}


\begin{frame}{Java und die JVM}
    Java, Kotlin, Clojure, Scala

\end{frame}

\begin{frame}{C\# und das .NET Framework}

\end{frame}



% ==============================================================================
% AB HIER FOLIEN EINFÜGEN
% ==============================================================================



\end{document}